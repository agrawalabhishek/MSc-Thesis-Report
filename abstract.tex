\chapter*{Abstract}
\setheader{Abstract}
\addcontentsline{toc}{section}{Abstract}

We study the orbital motion of regolith around asteroids, lofted from the surface due to impact cratering events, to understand the displacement of material on the surface and in orbit. The cratering events could be natural such as from meteoroid impacts, or they can be induced from spacecraft activities such as in-situ sample collection. Understanding the dynamics of orbiting regolith is important for future science missions and commercial activities on asteroids. Knowledge about expected particulate environment due to impact ejecta can help mission designers in trajectory planning to avoid interference or damage from orbiting regolith with a spacecraft and/or its instruments. The same study could be exploited in the field of commercial in-situ asteroid mining for sorting material of different sizes and densities by artificially lofting them into an orbit.
\newline\newline
In this thesis, we aim towards understanding the orbital behavior and the final fate of lofted regolith in the presence of gravity and Solar perturbations. A complete numerical simulations approach is adopted to perform the study. We model the asteroid as a constant density triaxial ellipsoid and account for perturbing accelerations from Solar radiation pressure and the third-body effect of the Sun. Several particles are simulated to loft off the surface of an asteroid at varying velocities and in different directions. We conduct simulations for regoliths with four different area-to-mass ratios, lofted from three completely different locations on the asteroid. In addition to this, every simulation is further subjected to four different initial Solar phase angles. Thus, for each regolith type and for a given surface location, we create 4608 unique initial conditions. Each of these are numerically propagated until they meet one of the final fates, i.e., escape, capture or re-impact.
\newline\newline
We study the characteristic behavior of regolith in terms of the final fate they encounter, with and without Solar Perturbations. A link was found between the distribution of hyperbolic excess velocity for all regoliths that escaped and whether they took zero or multiple orbital revolutions around the asteroid before escaping. The re-impact locations were mapped for all regolith types and for all three loft locations. The most distinct separation in re-impact locations between different regolith types was observed when they were lofted from the longest edge of the ellipsoid shaped asteroid. The third category of final fate, i.e., temporary orbital capture, was observed for extremely few initial conditions and only when Solar perturbations were considered. It was observed that Solar perturbations, even in small magnitudes, created orbital phase differences relative to trajectory simulations that omitted them; which led to the formation of capture orbits. Finally, we tested if the varying orbital behavior of regolith of different area-to-mass ratios can be exploited, when lofted with the same initial conditions, for the application of asteroid mining in terms of passive material sorting.

%% Extra information

% Since regolith are small dust particles, the effect of Solar perturbations would be more pronounced for them due to a higher area-to-mass ratio relative to a spacecraft.

% We derived a novel algorithm to estimate a non-conservative guaranteed escape speed for regolith on the surface of an asteroid. The algorithm was intended to identify cases with initial loft velocities that would lead to an escape but were missed by the conservative method. However, the algorithm did not work as expected and even failed to identify guaranteed escape cases detected by the conservative method.
% \newline\newline
