Monte Carlo is, simply said, the approach of using random numbers to compute something which is not
random (\cite{robert2013monte}, 2013). Monte Carlo simulations involve running the same simulation a
large number of times with randomly determined inputs to obtain the probabilistic distribution of or
other information about a problem. Note the distinction between Monte Carlo methods and simulations:
simulation refers to producing random variables with a certain distribution just to look at the simulation results, whereas Monte Carlo methods are a tool of quantitative statistical analysis; these are beyond the scope of this text. Performing a Monte Carlo simulation is especially useful for problems with many degrees of freedom and nonlinearities, of which the distribution of the outcome cannot be estimated analytically.

 

@book{robert2013monte,
  title={Monte Carlo statistical methods},
  author={Robert, Christian and Casella, George},
  year={2013},
  publisher={Springer Science \& Business Media}
}