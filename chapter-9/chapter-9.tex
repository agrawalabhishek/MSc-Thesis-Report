\chapter{Conclusions}
\label{conclusion}

This literature study report focused on a thorough study of the dynamics of a binary asteroid system and of a spacecraft/particle orbiter around the binary asteroids. The study of a binary asteroid system was inspired by the joint mission between \gls{ESA} and \gls{NASA}, called \gls{AIDA}, which involves sending two spacecrafts to the binary asteroid system Didymos. This report provides an extensive coverage of the various models and methods available to simulate the gravitational potential of an asteroid in general. After a careful evaluation of all the methods covered in this report, the polyhedron model was chosen to simulate the gravity field of an asteroid as it offered higher accuracy and a more realistic representation of an irregularly shaped asteroid. A study of some basic perturbation models, such as third-body effects and Solar Radiation Pressure, was also conducted. In general, these perturbations are accounted for in the motion of an orbiter around an asteroid. We calculated the first-order values for the perturbing accelerations from the gravitational acceleration of the Sun and Jupiter, the Solar Radiation Pressure and compared it to the acceleration due to gravity of a reference asteroid 433 Eros. These perturbing accelerations were calculated for a single regolith particle. The first order perturbing accelerations were calculated to assess their relative significance. We found that the perturbing acceleration due to the third-body effect of the Sun and the Solar Radiation Pressure was of the same order of magnitude as the acceleration due to gravity of Eros. Hence, it was concluded that the perturbations due to the third-body effect of the Sun and Solar Radiation Pressure can not be ignored in the dynamics simulator of an orbiting particle or spacecraft around an asteroid. The equations of motion for the full two-body problem, i.e. the relative motion of two non-point mass asteroids around each other, was studied and presented in the report. The equations of motion were presented for different configurations of the binary system, such as sphere-ellipsoid, ellipsoid-ellipsoid and polyhedron-polyhedron. The equations of motion for a massless third-body orbiting around a binary asteroid system were also studied for the same three binary configurations mentioned earlier. To simulate the motion of a spacecraft or a particle around an asteroid, the differential equations of motion have to be integrated. The structure of these equations is such that it is difficult to integrate them analytically and so a numerical integrator is used instead. This report provides an extensive coverage on the various numerical integration schemes available in literature and compares their performance with one another. Every scheme has its own advantages and disadvantages and as of now, none are universal. For our application, which involves long simulation times, the extrapolation method is the most suitable scheme for numerical integration. This literature study report also presented a brief chapter on the Monte Carlo simulation technique; a method which is needed to observe the motion of several regolith particles, each with a different set of initial conditions, around an asteroid at once. The report concludes by providing a rudimentary introduction to the Dynamical Systems Theory and its role in characterizing orbital motion in general.

It is emphasized again that the literature study report began with a focus on the motion of a spacecraft around a binary asteroid system, but later on the focus was shifted to the study of regolith around an asteroid. The tentative research questions for the latter are presented in \Cref{intro}. The final research questions, along with the research timeline and corresponding work-packages will be presented in the thesis report that follows.
