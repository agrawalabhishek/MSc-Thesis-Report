\chapter{Recommendations for Future Work}
\label{chap:recommendations}
\graphicspath{{Recommendations/Images/}}

After concluding the work done in this thesis, we will now state a few recommendations for future work. These could be pursued in a new MSc thesis or perhaps even a PhD. Since the simulator \gls{NAOS} designed for this thesis is highly modular and well tested, all of the following recommendations can be implemented in any future work by easily adding the required components to the existing structure of the simulator.

\begin{itemize}
\item We used a \gls{CDE} gravity potential model, which can account for an irregular gravity field for an elongated body, however, more realistic results can be obtained by incorporating a polyhedron gravity model for the asteroid. Since the Hayabusa-2 mission is scheduled to arrive at its target asteroid Ryugu by mid-2018, which intends to use a carry-on impactor to extract sub-surface material for analysis; simulations for lofting particle from the location of impact can be conducted with both the \gls{CDE} and the polyhedron gravity model to compare the regolith re-impact results from the simulation with actual changes in regolith deposits on the asteroid.

\item The asteroid was assumed to be in uniform rotation about its shortest axis, i.e. the largest moment of inertia. We can assess the effect of an asteroid's spin state on regolith motion by accounting for a non-uniform spin state.

\item The asteroid was assumed to be in a circular planar orbit around the Sun. This simple setting did not hinder in understanding the effect of perturbations on regolith motion. The properties of the asteroid's heliocentric motion could have been easily varied in the current thesis but they weren't because the existing degrees-of-freedom in the simulator presented ample amounts of data to be analyzed. Thus, for future work, one can employ out of plane eccentric heliocentric orbits for the asteroid to guage its effect on regolith motion and final fate volume. The best way to do this would be by using the orbital parameters of asteroid Ryugu and compare the corresponding results with the ones in this thesis.

\item The lofted material was simulated in the shape of a cone, whose axis was aligned with the local surface normal. We can consider inclining the axis of the cone itself with the local normal, thereby adding another angular degree-of-freedom to launch azimuth and declination. In this way one can simulate the effect of oblique impacts that could loft regolith from the surface of the asteroid.

\item We considered a unary asteroid in our case, but a binary or ternary asteroid system can be considered in future to study material transfer between the primary and the secondary asteroids in the event of meteoroid impacts.

\item In this thesis, we have modeled the dynamics of the regolith around the asteroid as a perturbed two-body problem. We numerically integrated the equations of motion to observe the behavior and final fate of the particles around the asteroid. Another way to analyze the final fate behavior, could perhaps be through the use of Weak Stability Boundaries. The theory was developed mainly in the framework of low-energy orbital transfers but it may have potential in recognizing capture orbits around the asteroid. By using the algorithmic definition of weak stability boundaries as described by \cite{garcia2007WSB}, the stable and unstable regions around the asteroid can be found by varying the initial velocity and the eccentricity of the initial osculating planar elliptical orbit around the asteroid. The particle is given the initial velocity at the pericentre of the initial osculating ellipse and hence the particle is not originating from the surface of the asteroid. However, we could loft particles from the surface of the asteroid and correlate their motion through the stable/unstable regions with the corresponding final fates.

\item A large volume simulation can be conducted by varying the initial conditions such as the Solar phase angle, launch declination, launch azimuth, launch location and velocity, at a much finer resolution. The data generated from such a simulation would be enormous and it wouldn't be possible to do a piece-by-piece analysis like we did in this report. However, one can use deep learning or machine learning algorithms to mine through the (big) data in order to hunt for any patterns associated between initial state and the corresponding final fate. This analysis can be done with the existing simulation models without making any changes in them.
\end{itemize}

The aforementioned list is by no means an exhaustive one, but they are good starting points from which more detailed recommendations can be derived to further improve our understanding on the orbital motion of regolith around asteroids.
