\chapter{Research Questions \& Goals}
\label{chap:research_questions}
\graphicspath{{Research_Questions/Images/}}

The study of the dynamics of a particle, on or around an asteroid, can be broadly divided into three main regimes. The first regime involves the study of surface ejecta generation, from natural events such as interplanetary particle impacts, cratering by other asteroids \& electrostatic dust levitation, or from space exploration events where the natural state of the regolith is disturbed by spacecraft sampling activities. The second regime involves the study of the subsequent orbital behavior of impact ejecta or lofted regolith under varying parameters such as launch conditions, asteroid rotational state \& shape, regolith particle size and density, Solar phase etc. And finally, the third regime involves the study of particle dynamics when it re-impacts with the surface of the asteroid. This thesis will concern itself with the second regime of research, i.e., the natural orbital evolution of regolith lofted from an asteroid’s surface.
%
\newline\newline
%
As mentioned earlier in \Cref{chap:intro}, understanding particulate environment around small-bodies has been identified by \gls{NASA} as a strategic knowledge gap. Understanding and developing tools or knowledge to estimate the orbital behavior and final fate of lofted regolith with greater accuracy is important for future space exploration missions (see \Cref{sec:future_missions}) that will involve direct interactions with asteroids, to avoid any damage to the spacecraft or surface robotic crew from orbiting particles. High-fidelity simulations of particulate motion can also help scientists in understanding the surface morphology of asteroids by helping them recreate cratering events. In \Cref{sec:literature_review}, we highlighted the shortfalls in the research done on the topic so far and we identified a gap that needs to be filled, and hence, the following top level research question is set:
\vspace{5mm}
%%%
\begin{center}
    \fbox{\parbox{0.8\textwidth}{
    \centering
    \textbf{\textit{Can we explain the orbital behavior and eventual fate of lofted regolith around an asteroid in presence of gravity and Solar perturbations?}}}
    }
\end{center}
%%%
\vspace{5mm}
This top-level research question is divided into the following sub-questions that help in structuring the thesis:
%%%
\begin{enumerate}
\item Does the regolith, launched from different locations such as leading, trailing, longest and shortest edge of an asteroid, show characteristic differences with regard to its final fate?
\item Can clear demarcation be established between the re-impact, capture, and escape scenarios, for the lofted regolith, based solely on the initial conditions?
\item What causes the regolith to enter into a temporary capture orbit around the asteroid?
\item For the same launch conditions, how does the orbital behavior and final fate of the regolith differ for different particle sizes and densities?
\item For the same particle size and density, how does the orbital behavior and final fate change with different launch locations?
\item Can we establish a non-conservative analytical expression to determine guarantee escape speed in presence of perturbations?
\item Can we exploit the orbital behavior of lofted regolith for sorting material of different sizes and densities as an application for asteroid mining?
\end{enumerate}
%%%
\vspace{5mm}
In order to answer these questions, the following main research goal is set:
%%%
\begin{center}
    \fbox{\parbox{0.8\textwidth}{
    \centering
    \textbf{\textit{Investigate the orbital motion of regolith launched from the surface of an asteroid using numerical simulations.}}}
    }
\end{center}
%%%
The sub-research goals are mentioned as follows:
%%%
\begin{enumerate}
\item Develop a modular and robust software tool that can propagate the trajectory of spherical particles around an asteroid for given initial conditions.
\item Develop software tools to plot and analyze numerical simulation results
\item Validate the software tools.
\item Perform simulations for particles launched from the asteroid's surface with different initial conditions, launch locations, and for different particle sizes \& densities.
\item Perform qualitative and quantitative analysis on numerical simulation results.
\item Document results and inferences for thesis report and peer reviewed journal paper.
\end{enumerate}
%%%
The vast majority of the time will be spent on designing the simulator and data processing \& visualization tools (see \Cref{chap:naos}), followed by their verification and validation (see \Cref{chap:v_and_v}). A relatively smaller time would then remain to perform the research and investigate the results, however the time remaining for this would be sufficient to answer all our research questions.
